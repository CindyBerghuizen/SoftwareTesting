\documentclass{article}

\usepackage{color}
\usepackage{listings}
\usepackage{fullpage}
\usepackage{graphicx}

\definecolor{gray_ulisses}{gray}{0.55}
\definecolor{castanho_ulisses}{rgb}{0.71,0.33,0.14}
\definecolor{preto_ulisses}{rgb}{0.41,0.20,0.04}
\definecolor{green_ulises}{rgb}{0.2,0.75,0}

\lstdefinelanguage{HaskellUlisses}
{
	basicstyle=\ttfamily\scriptsize,
	%backgroundcolor=\color{yellow},
	%frameshape={RYRYNYYYY}{yny}{yny}{RYRYNYYYY}, %contornos... muito nice...
	sensitive=true,
	morecomment=[l][\color{gray_ulisses}\scriptsize]{--},
	morecomment=[s][\color{gray_ulisses}\scriptsize]{\{-}{-\}},
	morestring=[b]",
	stringstyle=\color{red},
	showstringspaces=false,
	numbers=none,
	firstnumber=\thelstnumber,
	numberstyle=\tiny,
	numberblanklines=true,
	showspaces=false,
	showtabs=false,
	xleftmargin=15pt,
	xrightmargin=-20pt,
	emph=
	{[1]
		FilePath,IOError,abs,acos,acosh,all,and,any,appendFile,approxRational,asTypeOf,asin,
		asinh,atan,atan2,atanh,basicIORun,break,catch,ceiling,chr,compare,concat,concatMap,
		const,cos,cosh,curry,cycle,decodeFloat,denominator,digitToInt,div,divMod,drop,
		dropWhile,either,elem,encodeFloat,enumFrom,enumFromThen,enumFromThenTo,enumFromTo,
		error,even,exp,exponent,fail,filter,flip,floatDigits,floatRadix,floatRange,floor,
		fmap,foldl,foldl1,foldr,foldr1,fromDouble,fromEnum,fromInt,fromInteger,fromIntegral,
		fromRational,fst,gcd,getChar,getContents,getLine,head,id,inRange,index,init,intToDigit,
		interact,ioError,isAlpha,isAlphaNum,isAscii,isControl,isDenormalized,isDigit,isHexDigit,
		isIEEE,isInfinite,isLower,isNaN,isNegativeZero,isOctDigit,isPrint,isSpace,isUpper,iterate,
		last,lcm,length,lex,lexDigits,lexLitChar,lines,log,logBase,lookup,map,mapM,mapM_,max,
		maxBound,maximum,maybe,min,minBound,minimum,mod,negate,not,notElem,null,numerator,odd,
		or,ord,otherwise,pi,pred,primExitWith,print,product,properFraction,putChar,putStr,putStrLn,quot,
		quotRem,range,rangeSize,read,readDec,readFile,readFloat,readHex,readIO,readInt,readList,readLitChar,
		readLn,readOct,readParen,readSigned,reads,readsPrec,realToFrac,recip,rem,repeat,replicate,return,
		reverse,round,scaleFloat,scanl,scanl1,scanr,scanr1,seq,sequence,sequence_,show,showChar,showInt,
		showList,showLitChar,showParen,showSigned,showString,shows,showsPrec,significand,signum,sin,
		sinh,snd,span,splitAt,sqrt,subtract,succ,sum,tail,take,takeWhile,tan,tanh,threadToIOResult,toEnum,
		toInt,toInteger,toLower,toRational,toUpper,truncate,uncurry,undefined,unlines,until,unwords,unzip,
		unzip3,userError,words,writeFile,zip,zip3,zipWith,zipWith3,Impl,Equiv,Prop,Neg,Cnj,Dsj
	},
	emphstyle={[1]\color{blue}},
	emph=
	{[2]
		Bool,Char,Double,Either,Float,IO,Integer,Int,Maybe,Ordering,Rational,Ratio,ReadS,ShowS,String,NoTriangle,Equilateral,Rectangular,Isosceles,Other,Shape
	},
	emphstyle={[2]\color{castanho_ulisses}},
	emph=
	{[3]
		case,class,data,deriving,do,else,if,import,in,infixl,infixr,instance,let,
		module,of,primitive,then,type,where
	},
	emphstyle={[3]\color{preto_ulisses}\textbf},
	emph=
	{[4]
		quot,rem,div,mod,elem,notElem,seq
	},
	emphstyle={[4]\color{castanho_ulisses}\textbf},
	emph=
	{[5]
		EQ,False,GT,Just,LT,Left,Nothing,Right,True,Show,Eq,Ord,Num
	},
	emphstyle={[5]\color{preto_ulisses}\textbf}
}

\lstnewenvironment{code}
{\lstset{language=HaskellUlisses}}
{\smallskip}


\begin{document}
\setlength{\parindent}{0cm}

\title{Software Testing Assignment 4}
\author{Cindy Berghuizen, Omar Pakker , Chiel Peter, Maria Gouseti}
\date{\today}
\maketitle
\section*{1: Book Exercise}
-
\section*{2: Random data generator}

\begin{code}
genSetMax = 100
genSetMaxEntries = 10

genSet :: IO (Set Int)
genSet =  do
   n  <- getRandomInt genSetMaxEntries
   ns <- genSet' genSetMax n
   return ns

genSet' :: (Eq a, Num a) => Int -> a -> IO (Set Int)
genSet' _ 0 = return (Set [])
genSet' d c = do
         n <- getRandomInt d
         ns <- genSet' d (c-1)
         return (insertSet n ns)

\end{code}
Time spent: 10 minutes

\section*{Set intersectiom, union, difference}

\subsection*{Haskell Program}
\begin{code}
intersectSet :: (Ord a) => Set a -> Set a -> Set a 
intersectSet (Set [])     set2  =  (Set [])
intersectSet (Set (x:xs)) set2  | inSet x set2 = insertSet x (intersectSet (Set xs) set2)
                                | otherwise = (intersectSet (Set xs) set2)

-- Pre: no duplicates, it is sorted
-- Post: no duplicates, it is sorted 
differenceSet :: (Ord a) => Set a -> Set a -> Set a
differenceSet (Set [])     set2  =  (Set [])
differenceSet (Set (x:xs)) set2  | inSet x set2 = (differenceSet (Set xs) set2)
                                 | otherwise = insertSet x (differenceSet (Set xs) set2)                             

--Union is already implemented in SetOrd.hs
\end{code}

\subsection*{Intersection Test}
\begin{code}
-- A set I is an intersection of A and B if I is an subset of A and B 
testIntersect :: (Ord a) => Set a -> Set a -> Set a -> Bool
testIntersect a b i = subSet i a && subSet i b

automatedI :: IO Bool
automatedI = do
    a <- genSet
    b <- genSet
    return $ testIntersect a b (intersectSet a b)

automatedI' :: Int -> IO [Bool]
automatedI' 0 = return []
automatedI' c = do
     d <- automatedI
     ds <- automatedI'(c-1)
     return (d:ds)
     
generateIntersectionTest :: Int -> IO String
generateIntersectionTest c = do
    ps <- automatedI' c
    return ("All Checks Valid: " ++ (show (all (\x -> x) ps)))
\end{code}

\subsection*{Difference Test}
\begin{code}
-- Property: An set D is the difference of A and B if it is an subset
--  of A and has no elements in common with B                        
testDifference :: (Ord a) => Set a -> Set a -> Set a -> Bool
testDifference a b d = subSet d a && noElement d b

noElement :: (Ord a) => Set a -> Set a -> Bool
noElement (Set[]) _ = True
noElement (Set(x:xs)) set | inSet x set = False
                          | otherwise = noElement (Set xs) set
                          
automatedD :: IO Bool
automatedD = do
    a <- genSet
    b <- genSet
    return $ testDifference a b (differenceSet a b)

automatedD' :: Int -> IO [Bool]
automatedD' 0 = return []
automatedD' c = do
     d <- automatedD
     ds <- automatedD'(c-1)
     return (d:ds)
     
generateDifferenceTest :: Int -> IO String
generateDifferenceTest c = do
    ps <- automatedD' c
    return ("All Checks Valid: " ++ (show (all (\x -> x) ps)))
\end{code}

\subsection*{Union Test}
\begin{code}
--PROPERTY : Every element in either of the sets should be an element of the union
testUnion :: Int -> IO [Bool]
testUnion 0 = return []
testUnion a = do 	n <- testUnion1
			ns <- testUnion (a-1)			
			return(n : ns)

testUnion1 :: IO Bool
testUnion1 = do 	n <- randomIntSet
			m <- randomIntSet
			return(isElementOf n m (unionSet n m)) 

isElementOf :: (Ord a) => Set a -> Set a -> Set a -> Bool
isElementOf (Set a) (Set b) (Set c) = all (\x -> elem x c) (a++b)
\end{code}

\includegraphics{knipsel}

\newpage
Time spent: 75 minutes

\section*{Transitive Closure}
\begin{code}
trClos :: (Ord a) => Rel a -> Rel a
trClos x = trClos2 x []

-- If the closure n+1 is a subset of n than all closures are found
-- because no new elements were found
trClos2 :: (Ord a) => Rel a -> Rel a -> Rel a
trClos2  = fix (\ f x y ->
           if subSet (list2set x) (list2set y) then (nub x)
           else f ((x @@ x)++x) x) 
\end{code}

Time spent: 1.5 hours

\section*{Testing Closure}
\begin{code}
testTrClos :: (Ord a) => Rel a -> Rel a -> Bool
testTrClos [] _ = True
testTrClos (x:xs) z = (all (\y -> elem y z) ([x] @@ z)) && (testTrClos xs z)

--RANDOM TESTING
randomTestsTrClos :: Int -> IO [Bool]
randomTestsTrClos 0 = return []
randomTestsTrClos a = do 	n <- randomTestTrClos
				m <- randomTestsTrClos (a-1)
				return (n:m)

randomTestTrClos :: IO Bool
randomTestTrClos = do 	n <- randomRelation maxSize
			return( testTrClos (trClos n) (trClos n)) 

randomRelation :: (Eq a, Num a) => a -> IO [(Int,Int)]
randomRelation 0 = return []
randomRelation a = do 	n <- getRandomInt range
			m <- getRandomInt range
			l <- randomRelation (a-1)
			return((n,m):l) 

\end{code}

\includegraphics{knipsel2}

Time spent: 2 hours

\end{document}