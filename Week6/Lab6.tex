\documentclass{article}

\usepackage{color}
\usepackage{listings}
\usepackage{fullpage}
\usepackage{graphicx}
\usepackage{hyperref}

\definecolor{gray_ulisses}{gray}{0.55}
\definecolor{castanho_ulisses}{rgb}{0.71,0.33,0.14}
\definecolor{preto_ulisses}{rgb}{0.41,0.20,0.04}
\definecolor{green_ulises}{rgb}{0.2,0.75,0}

\lstdefinelanguage{HaskellUlisses}
{
	basicstyle=\ttfamily\scriptsize,
	%backgroundcolor=\color{yellow},
	%frameshape={RYRYNYYYY}{yny}{yny}{RYRYNYYYY}, %contornos... muito nice...
	sensitive=true,
	morecomment=[l][\color{gray_ulisses}\scriptsize]{--},
	morecomment=[s][\color{gray_ulisses}\scriptsize]{\{-}{-\}},
	morestring=[b]",
	stringstyle=\color{red},
	showstringspaces=false,
	numbers=none,
	firstnumber=\thelstnumber,
	numberstyle=\tiny,
	numberblanklines=true,
	showspaces=false,
	showtabs=false,
	xleftmargin=15pt,
	xrightmargin=-20pt,
	emph=
	{[1]
		FilePath,IOError,abs,acos,acosh,all,and,any,appendFile,approxRational,asTypeOf,asin,
		asinh,atan,atan2,atanh,basicIORun,break,catch,ceiling,chr,compare,concat,concatMap,
		const,cos,cosh,curry,cycle,decodeFloat,denominator,digitToInt,div,divMod,drop,
		dropWhile,either,elem,encodeFloat,enumFrom,enumFromThen,enumFromThenTo,enumFromTo,
		error,even,exp,exponent,fail,filter,flip,floatDigits,floatRadix,floatRange,floor,
		fmap,foldl,foldl1,foldr,foldr1,fromDouble,fromEnum,fromInt,fromInteger,fromIntegral,
		fromRational,fst,gcd,getChar,getContents,getLine,head,id,inRange,index,init,intToDigit,
		interact,ioError,isAlpha,isAlphaNum,isAscii,isControl,isDenormalized,isDigit,isHexDigit,
		isIEEE,isInfinite,isLower,isNaN,isNegativeZero,isOctDigit,isPrint,isSpace,isUpper,iterate,
		last,lcm,length,lex,lexDigits,lexLitChar,lines,log,logBase,lookup,map,mapM,mapM_,max,
		maxBound,maximum,maybe,min,minBound,minimum,mod,negate,not,notElem,null,numerator,odd,
		or,ord,otherwise,pi,pred,primExitWith,print,product,properFraction,putChar,putStr,putStrLn,quot,
		quotRem,range,rangeSize,read,readDec,readFile,readFloat,readHex,readIO,readInt,readList,readLitChar,
		readLn,readOct,readParen,readSigned,reads,readsPrec,realToFrac,recip,rem,repeat,replicate,return,
		reverse,round,scaleFloat,scanl,scanl1,scanr,scanr1,seq,sequence,sequence_,show,showChar,showInt,
		showList,showLitChar,showParen,showSigned,showString,shows,showsPrec,significand,signum,sin,
		sinh,snd,span,splitAt,sqrt,subtract,succ,sum,tail,take,takeWhile,tan,tanh,threadToIOResult,toEnum,
		toInt,toInteger,toLower,toRational,toUpper,truncate,uncurry,undefined,unlines,until,unwords,unzip,
		unzip3,userError,words,writeFile,zip,zip3,zipWith,zipWith3,Impl,Equiv,Prop,Neg,Cnj,Dsj
	},
	emphstyle={[1]\color{blue}},
	emph=
	{[2]
		Bool,Char,Double,Either,Float,IO,Integer,Int,Maybe,Ordering,Rational,Ratio,ReadS,ShowS,String,NoTriangle,Equilateral,Rectangular,Isosceles,Other,Shape
	},
	emphstyle={[2]\color{castanho_ulisses}},
	emph=
	{[3]
		case,class,data,deriving,do,else,if,import,in,infixl,infixr,instance,let,
		module,of,primitive,then,type,where
	},
	emphstyle={[3]\color{preto_ulisses}\textbf},
	emph=
	{[4]
		quot,rem,div,mod,elem,notElem,seq
	},
	emphstyle={[4]\color{castanho_ulisses}\textbf},
	emph=
	{[5]
		EQ,False,GT,Just,LT,Left,Nothing,Right,True,Show,Eq,Ord,Num
	},
	emphstyle={[5]\color{preto_ulisses}\textbf}
}

\lstnewenvironment{code}
{\lstset{language=HaskellUlisses}}
{\smallskip}


\begin{document}
\setlength{\parindent}{0cm}
\title{Software Testing Assignment 6}
\author{Cindy Berghuizen, Omar Pakker, Chiel Peter,  Maria Gouseti}
\date{\today}
\maketitle

\section*{Exercise 4}

\lstinputlisting[language=HaskellUlisses, firstline=67, lastline=76]{Lab6.hs}

k = 1, k =2 and k = 3 give 4 as a prime number. When the value of k gets higher there are less fool primes found, this is because more different random numbers are chosen for \emph{a} which lowers the probability a composite number is considered a prime. 

\vspace{0.5cm}
\section*{Exercise 5}

\lstinputlisting[language=HaskellUlisses, firstline=79, lastline=88]{Lab6.hs}

Carmichael numbers almost always pass the Fermat's primality check. That was also shown in the testing, most of the numbers passed our test.
 
The Carmichael numbers are of the form $ b^{n}\equiv b\pmod{n} $ for all integers $1<b<n-1 $. This is also how Fermat's little theorem define prime numbers ( $a^{p-1} \equiv 1 \pmod p.$) . Because Fermat defines prime numbers in the same way Carmichael defines the Carmichael numbers, the carmichael numbers do satisfy the definition of a prime number used in Fermat's primality check. Although the carmichael numbers are not prime numbers but do satisfy Fermat's definition of a prime number, they pass the testing.
\vspace{0.5cm}

\section*{Exercise 6}
\lstinputlisting[language=HaskellUlisses, firstline=91, lastline=100]{Lab6.hs}

Although some Carmichael numbers still pass the Miller-Rabin primality, these are significantly less than with Fermat's primality check. If we higher \emph{k}, meaning that we check with more random \emph{a's} we even find that Miller-Rabin doesn't consider any Carmichael numbers as prime numbers.

\lstinputlisting[language=HaskellUlisses, firstline=103, lastline=111]{Lab6.hs}
\begin{verbatim}
*Lab6> lengthCar 500 10
"500"
\end{verbatim}

\begin{verbatim}
*Lab6>lengthMR 500 10
"0"
\end{verbatim}

\section*{Exercise 7}
\lstinputlisting[language=HaskellUlisses, firstline=115, lastline=131]{Lab6.hs}

The numbers that give True for the first argument in the tuple are indeed known Mersenne numbers as can be found on \url{http://en.wikipedia.org/wiki/Mersenne_prime}.

\end{document}